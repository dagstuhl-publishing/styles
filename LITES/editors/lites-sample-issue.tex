
%This is a template for producing LIPIcs conference volumes.
%The usage of this file together with lipicsmaster.cls should be
%straightforward. There is no separate documentation.
%Feel free to adapt this file or lipicsmaster.cls if required.

\documentclass[a4paper,UKenglish]{litesmaster}
  %for A4 paper format use option "a4paper", for US-letter use option "letterpaper"
  %for british hyphenation rules use option "UKenglish", for american hyphenation rules use option "USenglish"

\usepackage{microtype}%if unwanted, comment out or use option "draft"
\usepackage{wrapfig}


%\graphicspath{{./graphics/}}%helpful if your graphic files are in another directory

\bibliographystyle{plain}% the recommnded bibstyle

\Volume{00}
\Issue{00}
\DatePublished{June 2010}
\Copyright{Editor} % copyright owner for foreword



\begin{document}

\frontmatter

\maketitle

\begin{publicationinfo}%for page ii, please fill as required
\sffamily
\twocolumn

{\Large\bf\sffamily \textbf{\href{http://www.dagstuhl.de/lites}{ISSN \printISSN{}}}}

\bigskip

\emph{Published online and open access by}\newline
the European Design and Automation Association (EDAA) /\ EMbedded Systems Special Interest Group (EMSIG) and
Schloss Dagstuhl -- Leibniz-Zentrum f\"ur Informatik GmbH, Dagstuhl Publishing, Saarbr\"ucken/Wadern, Germany. 

Online available at \\ \url{http://www.dagstuhl.de/dagpub/\printISSN}.

\bigskip
\emph{Publication date}\newline
\printDatePublished{}



\bigskip

\emph{Bibliographic information published by the Deutsche Nationalbibliothek}\newline
The Deutsche Nationalbibliothek lists this publication in the Deutsche Nationalbibliografie; detailed bibliographic data are available in the Internet at \href{http://dnb.d-nb.de}{http://dnb.d-nb.de}. 

\bigskip

\emph{License}\newline
This work is licensed under a Creative Commons Attribution 3.0 Germany license (CC BY~3.0~DE): \href{http://creativecommons.org/licenses/by/3.0/de/deed.en}{\nolinkurl{http://creativecommons.org/licenses/by/}}\linebreak \href{http://creativecommons.org/licenses/by/3.0/de/deed.en}{\nolinkurl{3.0/de/deed.en}}.
\begin{wrapfigure}[2]{l}{1.8cm}
\vspace*{-1\baselineskip}
\includegraphics[width=1.8cm]{cc-by}
\end{wrapfigure} 
In brief, this license authorizes each and everybody to share (to
copy, distribute and transmit) the work under the following
conditions, without impairing or restricting the authors'
moral rights:
\begin{itemize}
\item Attribution: The work must be attributed to its authors.
\end{itemize}

The copyright is retained by the corresponding authors.

%\bigskip
\vfill
\emph{Digital Object Identifier}\newline
\printIssueDOI

\newpage

\vphantom{{\Large\bf\sffamily \textbf{\href{http://www.dagstuhl.de/lites}{ISSN \printISSN{}}}}}

\bigskip

\emph{Aims and Scope}\newline
LITES aims at the publication of high-quality scholarly articles, ensuring efficient submission, reviewing, and publishing procedures. All articles are published open access, i.e., accessible online without any costs. The rights are retained by the author(s).

\medskip

LITES publishes original articles on all aspects of embedded computer systems, in particular: the design, the implementation, the verification, and the testing of embedded hardware and software systems; the theoretical foundations; single-core, multi-processor, and networked architectures and their energy consumption and predictability properties; reliability and fault tolerance; security properties; and on applications in the avionics, the automotive, the telecommunication, the medical, and the production domains. 

\bigskip

\emph{Editorial Board}
\begin{itemize}
\item Alan Burns (Editor-in-Chief)%, University of York)
\item Bashir Al Hashimi %(University of Southampton)
\item Karl-Erik Arzen %(Lund University)
\item Neil Audsley %(University of York)
\item Sanjoy Baruah %(University of North Carolina at Chapel Hill)
\item Samarjit Chakraborty %(Technical University Munich)
\item Marco di Natale %(Scuola Superiore Sant'Anna)
\item Martin Fr\"anzle %(Carl von Ossietzky University Oldenburg)
\item Steve Goddard %(University of Nebraska-Lincoln) 
\item Gernot Heiser% (University of New South Wales)
\item Axel Jantsch %(Royal Institute of Technology Stockholm)
\item Florence Maraninchi %(VERIMAG)
\item Sang Lyul Min %(Seoul National University)
\item Lothar Thiele %(ETH Z\"urich)
\item Mateo Valero %(Technical University of Catalonia)
\item Virginie Wiels %(ONERA)
\end{itemize}

\bigskip
\emph{Editorial Office}\newline
Michael Wagner \emph{(Managing Editor)}\\
Marc Herbstritt \emph{(Managing Editor)}\\
Jutka Gasiorowski \emph{(Editorial Assistance)}\\
Thomas Schillo \emph{(Technical Assistance)}

\bigskip
\emph{Contact}\newline
Schloss Dagstuhl -- Leibniz-Zentrum f\"ur Informatik\\
LITES, Editorial Office\\
Oktavie-Allee, 66687 Wadern, Germany\\ 
lites@dagstuhl.de


\vfill

\url{http://www.dagstuhl.de/lites}
 
 \thispagestyle{empty}
 \onecolumn
\end{publicationinfo}


\begin{contentslist}
%To generate the table of contents copy all the .vtc files
%of the contributions to your working directory.
%For every contribution type a line
\inputtocentry{dummycontribution}
%where the argument of \inputtocentry is the name of
%the vtc file without suffix.

%Alternatively write e.g.
\contitem
\title{Dummy title}
\author{John Q. Public}
\page{77}

%\part{} %use if volume is divided in parts
\end{contentslist}


\end{document}
