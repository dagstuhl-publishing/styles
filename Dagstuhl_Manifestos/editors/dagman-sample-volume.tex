%This is a template for producing a dagstuhl manifesto frontmatter.
%The usage of this file together with dagman-master.cls should be
%straightforward. There is no separate documentation.
%Feel free to adapt this file or dagman-master.cls if required.

\documentclass[a4paper,UKenglish]{dagman-master}
  %for A4 paper format use option "a4paper", for US-letter use option "letterpaper"
  %for british hyphenation rules use option "UKenglish", for american hyphenation rules use option "USenglish"

\usepackage{microtype}%if unwanted, comment out or use option "draft"
\usepackage{wrapfig}

%\graphicspath{{./graphics/}}%helpful if your graphic files are in another directory

\bibliographystyle{plain}% the recommnded bibstyle


\title{\includegraphics[width=0.67\textwidth]{dagman-logo-color}}

\subtitle{Volume XY, Issue Z, January -- December 2010}

\begin{document}

\maketitle

\begin{contentslist}
%To generate the table of contents copy all the .vtc files
%of the contributions to your working directory.
%For every contribution type a line
\inputtocentry{dummycontribution}
%where the argument of \inputtocentry is the name of
%the vtc file without suffix.

%Alternatively write e.g.
\contitem
\title{Dummy title}
\author{John Q. Public}
\page{77}

%\part{} %use if volume is divided in parts
\end{contentslist}


\bigskip

\vfill

\titlepagebottomline{Dagstuhl Manifestos, Vol.~1, Issue 1 \qquad \qquad \qquad  \quad ISSN 2193-2433}
\bottomline

%\clearpage


%% editorial setup / creative commons / ...

\begin{publicationinfo}%for page ii, please fill as required
\sffamily
%\columnsep30pt
\twocolumn



{\Large\bf\sffamily \textbf{\href{http://www.dagstuhl.de/dagman}{ISSN 2193-2433}}}

\bigskip
\bigskip

\emph{Published online and open access by}\newline 
Schloss Dagstuhl -- Leibniz-Zentrum f\"ur Informatik GmbH, Dagstuhl
Publishing, Saarbr\"ucken/Wadern, Germany. 

Online available at \href{http://www.dagstuhl.de/dagman}{http://www.dagstuhl.de/dagman}

\bigskip
\emph{Publication date}\newline
March, 2012

\bigskip
\emph{Bibliographic information published by the Deutsche
  Nationalbibliothek}\newline 
The Deutsche Nationalbibliothek lists this publication in the Deutsche
Nationalbibliografie; detailed bibliographic data are available in the
Internet at \url{http://dnb.d-nb.de}.

\bigskip
\emph{License}
\iffalse
\begin{wrapfigure}{l}[3cm]{2cm}
\includegraphics[width=2cm]{cc-by-nc-nd}
\end{wrapfigure}
\fi

This work is licensed under  a Creative Commons
Attribution-NonCommercial-NoDerivs 3.0 Unported license:
\href{http://creativecommons.org/licenses/by-nc-nd/3.0/legalcode}{CC-BY-NC-ND}.

\begin{wrapfigure}[2]{l}{1.8cm}
\vspace*{-1\baselineskip}
\includegraphics[width=1.8cm]{cc-by-nc-nd}
\end{wrapfigure} 
In brief, this license authorizes each and everybody to share (to
copy, distribute and transmit) the work under the following
conditions, without impairing or restricting the authors'
moral rights:

\begin{itemize}
\item Attribution: The work must be attributed to its authors.
\item Noncommercial: The work may not be used for commercial purposes. 
\item No derivation: It is not allowed to alter or transform this work.
\end{itemize}

\smallskip

The copyright is retained by the corresponding authors.

\vfill

Digital Object Identifier: \href{http://dx.doi.org/10.4230/DagMan.1.1.i}{10.4230/DagMan.1.1.i}

\newpage

\vphantom{{\Large\bf\sffamily \textbf{\href{http://www.dagstuhl.de/dagrep}{ISSN 2193-2433}}}}

\bigskip
\bigskip


\emph{Aims and Scope}\newline
The manifestos from Dagstuhl Perspectives Workshops are published in the \emph{Dagstuhl Manifestos} journal. Each manifesto aims for describing the state-of-the-art in a field along with its shortcomings and strenghts. Based on this, position statements and perspectives for the future are illustrated. A manifesto typically has a less technical character; instead it provides guidelines and roadmaps for a sustainable organisation of future progress. 

 
\bigskip
\bigskip
\emph{Editorial Board}
\begin{itemize}
\item Susanne Albers
\item Bernd Becker
\item Karsten Berns
\item Stephan Diehl
\item Hannes Hartenstein
\item Frank Leymann
\item Stephan Merz
\item Bernhard Nebel
\item Han La Poutr\'{e}
\item Bernt Schiele
\item Nicole Schweikardt
\item Raimund Seidel
\item Gerhard Weikum
\item Reinhard Wilhelm (\emph{Editor-in-Chief})
\end{itemize}

\bigskip
\emph{Editorial Office}\newline
Roswitha Bardohl \emph{(Managing Editor)}\\
Marc Herbstritt \emph{(Head of Editorial Office)}\\
Jutka Gasiorowski \emph{(Editorial Assistance)}\\
Thomas Schillo \emph{(Technical Assistance)}

\bigskip
\emph{Contact}\newline
Schloss Dagstuhl -- Leibniz-Zentrum f\"ur Informatik\\
Dagstuhl Manifestos, Editorial Office\\
Oktavie-Allee, 66687 Wadern, Germany\\ 
publishing@dagstuhl.de


\vfill

\href{http://www.dagstuhl.de/dagman}{www.dagstuhl.de/dagman}

  
\end{publicationinfo}


\end{document}
