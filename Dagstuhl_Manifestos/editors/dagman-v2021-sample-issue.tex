\documentclass[a4paper,USenglish]{dagman-master-v2021}
%for producing a PDF according the PDF/A standard, add "pdfa"

\pagestyle{empty}

\usepackage{microtype}%if unwanted, comment out or use option "draft"
\usepackage{wrapfig}
\usepackage{multicol}


\title{\includegraphics[width=0.67\textwidth]{dagman-logo-color}}

\ManifestosVolume{1}
\ManifestosIssue{1}
\ManifestosMonthYear{January 2011}

%\subtitle{Volume xx, Issue yy, January 2011} %use only to overwrite default

\begin{document}


\maketitle


\begin{contentslist}
%To generate the table of contents copy all the .vtc files
%of the contributions to your working directory.
%For every contribution type a line
\inputtocentry{dummycontribution}
%where the argument of \inputtocentry is the name of
%the vtc file without suffix.

%Alternatively write e.g.
\contitem
\title{Dummy title}
\author{John Q. Public}
\page{77}

%\part{} %use if volume is divided in parts
\end{contentslist}

\bigskip

\vfill

%\titlepagebottomline{Dagstuhl Manifestos, Vol.~x, Issue y \qquad \qquad \qquad \qquad \quad ISSN 2193-2433} %use only to overwrite default

\bottomline

%\clearpage


%% editorial setup / creative commons / ...

\begin{publicationinfo}%for page ii, please fill as required
\sffamily
%\columnsep30pt
\twocolumn


{\Large\bf\sffamily \textbf{\href{https://www.dagstuhl.de/dagman}{ISSN 2193-2433}}}

\bigskip
\bigskip

\emph{Published online and open access by}\newline 
Schloss Dagstuhl -- Leibniz-Zentrum f\"ur Informatik GmbH, Dagstuhl
Publishing, Saarbr\"ucken/Wadern, Germany.  Online available at\newline \href{https://www.dagstuhl.de/dagpub/2193-2433}{https://www.dagstuhl.de/dagpub/2193-2433}

\bigskip
\emph{Publication date}\newline
June, 2011

\bigskip
\emph{Bibliographic information published by the Deutsche
  Nationalbibliothek}\newline 
The Deutsche Nationalbibliothek lists this publication in the Deutsche
Nationalbibliografie; detailed bibliographic data are available in the
Internet at \url{https://dnb.d-nb.de}.

\bigskip
\emph{License}

This work is licensed under a \href{https://creativecommons.org/licenses/by/4.0/de/legalcode}{Creative Commons
Attribution 4.0 International license (CC BY 4.0)}.

\begin{wrapfigure}[2]{l}{1.8cm}
\vspace*{-1\baselineskip}
\includegraphics[width=1.8cm]{cc-by}
\end{wrapfigure} 
In brief, this license authorizes each and everybody to share (to
copy, distribute and transmit) the work under the following
conditions, without impairing or restricting the authors'
moral rights:
\begin{itemize}
\item Attribution: The work must be attributed to its authors.
\end{itemize}

\smallskip

The copyright is retained by the corresponding authors.

\vfill

Digital Object Identifier: \printDOI

\newpage

\vphantom{{\Large\bf\sffamily \textbf{\href{https://www.dagstuhl.de/dagman}{ISSN 2193-2433}}}}

\bigskip
\bigskip


\emph{Aims and Scope}\newline
The manifestos from Dagstuhl Perspectives Workshops are published in the \emph{Dagstuhl Manifestos} journal. Each manifesto aims for describing the state-of-the-art in a field along with its shortcomings, strengths. Based on this, position statements, perspectives for the future are illustrated. A manifesto typically has a less technical character; instead it provides guidelines, roadmaps for a sustainable organisation of future progress. 

\bigskip
\bigskip

%% The Editorial Board is equal to the Scientific Directorate (SD) as of the month to which the issue is related, e.g. Issue 10 (2044) => SD constitution as of October 2044
\emph{Editorial Board}
\begin{itemize}
\item Elisabeth Andr\'{e}
\item Franz Baader
\item Gilles Barthe
\item Daniel Cremers
\item Reiner H\"ahnle
\item Barbara Hammer
\item Lynda Hardman
\item Oliver Kohlbacher
\item Bernhard Mitschang
\item Albrecht Schmidt
\item Wolfgang Schr\"{o}der-Preikschat
\item Raimund Seidel (\emph{Editor-in-Chief})
\item Emanuel Thom\'{e}
\item Heike Wehrheim
\item Verena Wolf
\item Martina Zitterbart
\end{itemize}

\medskip
\emph{Editorial Office}\newline
Michael Wagner\emph{(Managing Editor)}\\
Jutka Gasiorowski \emph{(Editorial Assistance)}\\
Dagmar Glaser \emph{(Editorial Assistance)}\\
Thomas Schillo \emph{(Technical Assistance)}

\medskip
\emph{Contact}\newline
Schloss Dagstuhl -- Leibniz-Zentrum f\"ur Informatik\\
Dagstuhl Reports, Editorial Office\\
Oktavie-Allee, 66687 Wadern, Germany\\ 
reports@dagstuhl.de


\vfill

\href{https://www.dagstuhl.de/dagman}{https://www.dagstuhl.de/dagman}

  
\end{publicationinfo}


\end{document}