
%This is a template for producing a frontmatter of a TGDK ussze.
%The usage of this file together with tgdkmaster-v2021.cls should be
%straightforward. There is no separate documentation.
%Feel free to adapt this file or lipicsmaster.cls if required.

\documentclass[a4paper,UKenglish]{tgdkmaster-v2021}
  %for A4 paper format use option "a4paper", for US-letter use option "letterpaper"
  %for british hyphenation rules use option "UKenglish", for american hyphenation rules use option "USenglish"
  %for producing a PDF according the PDF/A standard, add "pdfa"

\usepackage{microtype}%if unwanted, comment out or use option "draft"
\usepackage{wrapfig}


%\graphicspath{{./graphics/}}%helpful if your graphic files are in another directory


%%%%%%%
% following macro block is one applicable for special issues
%
% only applicable if editors are responsible for complete issue, e.g. in case of a special issue
%\editor{John Q. Public}{Dummy University Computing Laboratory, [Address], Country}{johnqpublic@dummyuni.org}{https://orcid.org/0000-0002-1825-0097}%TODO in case of special issue, please use full name; only 1 editor per \editor macro; first two parameters are mandatory, other parameters can be empty.

%\editor{Joan R. Access}{School of Computer Science, University City, Country2}{beditor@uni-city.edu}{}

%\ccsdesc[100]{\textcolor{red}{Replace ccsdesc macro with valid one}}%TODO in case of special issue: Please choose ACM 2012 classifications from https://dl.acm.org/ccs/ccs_flat.cfm . 

%\EventEditors{John Q. Open and Joan R. Access} %editors of special issue
%\EventNoEds{2}
%\EventLongTitle{30th Conference on Very Important Topics (CVIT 2016)} %topic of special issue
%\EventShortTitle{CVIT 2016}
%\EventAcronym{CVIT}
%\EventYear{2016}
%\EventDate{December 24--27, 2016}
%\EventLocation{Little Whinging, United Kingdom}
%\EventLogo{}
%%%%%%%


\SeriesVolume{3}
\SeriesIssue{2}
\ArticleNo{0} % the frontmatter is always the first paper and has always the article number 0 (zero).

\DatePublished{November 2016}

\begin{document}

\frontmatter

\maketitle

\begin{publicationinfo}%for page ii, please fill as required
\sffamily
\twocolumn

{\Large\bf\sffamily \textbf{\href{https://www.dagstuhl.de/tgdk}{ISSN \printISSN{}}}}

\bigskip



\printEditorLong

\bigskip

\printSubjclass

\bigskip

\emph{Published online and open access by}\newline
Schloss Dagstuhl -- Leibniz-Zentrum f\"ur Informatik GmbH, Dagstuhl Publishing, Saarbr\"ucken/Wadern, Germany. 

Online available at \\ \url{https://www.dagstuhl.de/dagpub/\printISSN}.

\bigskip
\emph{Publication date}\newline
\printDatePublished{}



\bigskip

\emph{Bibliographic information published by the Deutsche Nationalbibliothek}\newline
The Deutsche Nationalbibliothek lists this publication in the Deutsche Nationalbibliografie; detailed bibliographic data are available in the Internet at \href{https://dnb.d-nb.de}{https://dnb.d-nb.de}. 

\bigskip

\emph{License}\newline
This work is licensed under a Creative Commons Attribution 4.0 International license (CC BY~4.0): \href{https://creativecommons.org/licenses/by/4.0/}{\nolinkurl{https://creativecommons.org/licenses/by/4.0}}.
\begin{wrapfigure}[2]{l}{1.8cm}
\vspace*{-1\baselineskip}
\includegraphics[width=1.8cm]{cc-by}
\end{wrapfigure} 
In brief, this license authorizes each and everybody to share (to
copy, distribute and transmit) the work under the following
conditions, without impairing or restricting the authors'
moral rights:
\begin{itemize}
\item Attribution: The work must be attributed to its authors.
\end{itemize}

The copyright is retained by the corresponding authors.

%\bigskip
\vfill
\emph{Digital Object Identifier}\newline
\printForewordDOI

\newpage

\vphantom{{\Large\bf\sffamily \textbf{\href{https://www.dagstuhl.de/tgdk}{ISSN \printISSN{}}}}}

\bigskip

\emph{Aims and Scope}\newline
Transactions on Graph Data and Knowledge (TGDK) is an Open Access journal that publishes original research articles and survey articles on graph-based abstractions for data and knowledge, and the techniques that such abstractions enable with respect to integration, querying, reasoning and learning. The scope of the journal thus intersects with areas such as Graph Algorithms, Graph Databases, Graph Representation Learning, Knowledge Graphs, Knowledge Representation, Linked Data and the Semantic Web. Also in-scope for the journal is research investigating graph-based abstractions of data and knowledge in the context of Data Integration, Data Science, Information Extraction, Information Retrieval, Machine Learning, Natural Language Processing, and the Web.

\medskip

The journal is Open Access without fees for readers nor for authors (also known as Diamond Open Access).	

\bigskip

\emph{Editors in Chief}
\begin{itemize}
\item Ian Horrocks %– University of Oxford, U.K.
\item Lalana Kagal %– Massachusetts Institute of Technology, U.S.
\item Andreas Hotho %– University of Würzburg, Germany
\item Aidan Hogan %– IMFD; DCC, University of Chile, Chile
\end{itemize}

\bigskip
\emph{Editorial Office}\newline
Schloss Dagstuhl -- Leibniz-Zentrum f\"ur Informatik\\
TGDK, Editorial Office\\
Oktavie-Allee, 66687 Wadern, Germany\\ 
tgdk@dagstuhl.de


\vfill

\url{https://www.dagstuhl.de/tgdk}
 
 \thispagestyle{empty}
 \onecolumn
\end{publicationinfo}


\begin{contentslist}
%To generate the table of contents copy all the .vtc files
%of the contributions to your working directory.
%For every contribution type a line
\inputtocentry{dummycontribution}
%where the argument of \inputtocentry is the name of
%the vtc file without suffix.

%Alternatively write e.g.
\contitem
\title{Dummy title}
\author{John Q. Public}
\page{77}

%\part{} %use if volume is divided in parts
\end{contentslist}

\chapter{Preface} %please fill or comment out

Lorem ipsum dolor sit amet, consectetur adipiscing elit. Ut mattis
elementum fermentum. Pellentesque habitant morbi tristique senectus et
netus et malesuada fames ac turpis egestas. Nulla sapien magna,
bibendum in dictum sed, egestas vel purus. Pellentesque id ornare
lacus. Pellentesque justo elit, sodales a fringilla vitae, gravida sed
elit. Etiam turpis eros, tincidunt sit amet tempor sed, gravida quis
eros. Mauris et nunc enim. Ut congue rhoncus odio vitae lacinia. Nunc
placerat est eu eros dignissim ac tristique nisi placerat. Fusce et
hendrerit justo. Nunc feugiat pulvinar nunc ac tincidunt. Donec eu
pharetra metus. Cras malesuada ante accumsan purus dignissim
euismod. Curabitur risus ante, aliquet ut suscipit eget, vulputate non
ligula. Nullam eleifend malesuada est, nec adipiscing sapien eleifend
eget. Vestibulum fringilla diam id felis sagittis aliquet. Maecenas
sed metus vel dui vulputate pretium et id lacus. Ut eget libero augue,
ut aliquet orci. Integer sed nunc id massa interdum imperdiet. Nunc ut
consequat eros.

Morbi hendrerit dapibus augue. Proin sed adipiscing ipsum. Ut
vulputate ultricies diam id dictum. Nunc pharetra imperdiet
sodales. Morbi convallis massa vitae justo adipiscing nec congue nunc
fringilla. Pellentesque eu rhoncus ligula. Nam eros neque, hendrerit a
rhoncus vel, molestie nec nulla. Ut iaculis vulputate mauris, non
scelerisque dolor fringilla sit amet. Pellentesque habitant morbi
tristique senectus et netus et malesuada fames ac turpis
egestas. Quisque vitae accumsan risus. Sed molestie dictum
venenatis. Sed odio justo, gravida et vulputate eu, congue et lorem. 

\begin{participants}
\chapter[Authors]{List of Authors}
%use \participant for every author, eg.:
\participant John Q. Public\\ 
  Dummy University Computing Laboratory\\
  Address, Country\\
  johnqpublic@dummyuni.org

\end{participants} 

\end{document}
